\chapter{Conclusion}

Many enterprise applications have requirements involving operations
that may span arbitrarily long periods of time. Yet, most modern data
persistence frameworks are lackluster in regard to Long Lived
Transactions. This forces programmers to devise clever ways to
implement their requirements, promoting bad engineering practices and
adding unnecessary complexity to the system.

The Fenix Framework, as a Framework to support enterprise applications
that require a persistent and transactional rich domain model, merely
provided support for regular (short-lived) transactions, and as such,
suffered from the same perils of many other frameworks.

This document has described an extension to the Fenix Framework that
allows drop-in support for Long Lived Transactions, without the need
to modify existing business code. The presented extension shows
many characteristics desirable of such a solution. These include
support for system restarts by storing intermediate data persistently,
support for concurrent users collaborating on a single Long Lived
Transaction, the same correctness guarantees as regular transactions
as well as comparable performance on the execution of the
transaction's steps.

~\\

Throughout this document I presented a description of Long Lived
Transactions, explained why they are difficult to implement using the
currently available tools and shown several approaches programmers use
to implement their requirements on top of short-lived transactions. I
presented previous work on Database Transactions, Long Lived
Transactions, Workflow Management Systems and Software Transactional
Memories, in an effort to understand the current solutions. I also
demonstrated that the currently existing solutions are unfit to the
problem I was trying to solve.

~\\

To familiarize the reader with the Fenix Framework, I provided a brief
description of the Framework's architecture, with detailed
descriptions of the parts most relevant to the solution.

I then presented my solution to the problem at hand, showing the
proposed architecture and implementation details, providing rationales
for every design decision.

The initial implementation, while fully functional, present very poor
performance results, rendering even the simplest of test cases
unusable. I then presented an in-depth analysis of the implementation,
and was able to track down the reasons for such poor performance. By
doing several improvements on how the intermediate data is stored and
searched, I was able to accomplish very good performance results,
providing execution times close to those of regular transactions.

~\\

There is, however, one major limitation in the proposed solution. To
ensure the correctness of the solution, transaction validation is
performed in a similar way to regular transactions, by ensuring that
the outcome of the transaction is not affected by reading outdated
values. Due to the duration of a Long Lived Transaction, the
probability of conflicts is much larger than with regular
transactions. This can be a problem, as a large amount of work can be
lost if a Long Lived Transaction is aborted. Even though solving this
limitation is out of the scope of this work, I presented some ideas
that could address this issue.