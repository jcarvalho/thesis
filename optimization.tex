\chapter{Optimization}

As shown in Chapter \ref{chap:validation}, the initial implementation,
while functional, presented very poor performance results. This
chapter describes the optimizations applied to the implementation,
resulting in remarkable practical results.

\section{Annotations}

Is this worth mentioning?

\section{Read-Set differentiation}

In the implementation described in Chapter \ref{chap:solution}, both
the Write-Set and the Read-Set were represented using {\it LogEntries}.

Recalling Figure \ref{fig:architecture}, {\it LogEntries} store a
reference to the respective DomainObject, the object's slot, and the
slot's value. These objects were used to store elements of both the
Read-Set and the Write-Set, which proved to be quite expensive, as the
Read-Set only cares about which slots were read, completely
disregarding its value.

As such, 


\section{Using BPlusTrees to hold LogEntries}

\section{Removing LogEntries}

\section{Fine tuning WriteSet}

