\chapter{Long-Lived Transactions}

\section{What are Long-Lived Transactions?}
\label{sec:what}

As previously described, Long-Lived Transactions are transactions with
a lifetime larger than a typical database transaction. To better
understand this concept, consider the example shown in
Figure~\ref{fig:courseDomain}, corresponding to a simplified fragment
of the domain model for an application in the higher education domain.

\begin{figure}
  \centering
  \begin{tikzpicture}

\begin{class}[text width=4.5cm]{Course}{0.5,0}
  \attribute{name : String} \attribute{objectives : String}
  \attribute{credits : int} \attribute{bibliography : Publication[]}
\end{class}

\begin{class}[text width=3cm]{Department}{7.5,-0.85}
\end{class}

\association{Department}{}{1}{Course}{}{*}

\end{tikzpicture}

\caption{Sample Domain Model in UML}
\label{fig:courseDomain}

\end{figure}

In this simplified domain model, a course belongs to a department, has
a name, its objectives, the credits granted upon completion, and the
recommended bibliography. The domain is deemed to be consistent only
if all attributes of course have a defined value, and each course must
have a department. Each department is responsible for managing its
courses, meaning that it is up to someone who works in the department
to start the process of creating new courses.

Note that the creation of a new course should be executed
transactionally, because other users of the system should not be able
to see a course in an inconsistent state (either not connected to any
departement and/or without all attributes properly defined).

The pseudo-code in Listing~\ref{fig:courseCreation} implements the
business operation of creating a new course. Assume that {\bf
  department} is inferred from the user performing the operation.

\begin{lstlisting}[float, caption={Pseudo-Code for the creation of a
    new course. A new Course is created, associated with its
    department, and its attributes are filled.},
  label={fig:courseCreation}]
Course course = new Course();
department.addCourse(course);
course.setName(name);
course.setObjectives(objectives);
course.setCredits(credits);
course.setBibliography(bibliography);
\end{lstlisting}

There are several ways to implement this operation. I will now
describe two common scenarios for said implementation.

\subsection{Business Transaction in a single interaction}

A possible implementation (most likely, the simplest) of the course
creation operation in a web application is to have a single page in
which the user provides all the required information. Once the
information is submitted, a new Course is created, associated with the
department of the person performing the operation, and all its
attributes are filled according to the information submitted by the
user. The new object is then stored in the database, making it
persistent and available for other users to view.

In this scenario, in which all the information can be provided in a
single user interaction, the transactional guarantees of the operation
are ensured by the underlying database (which is assumed to provide
the classic transactional semantics \cite{gray1981transaction}),
because the whole operation can be performed within the scope of a
single database transaction.

An important consequence of implementing the operation in a single
database transaction is that the programmer can manipulate all the
domain objects involved directly. The order in which the modifications
to the objects are performed is irrelevant, as long as the domain is
consistent when the transaction is committed. This is the semantics
typically expected by a programmer of such applications: there may be
instants in which some domain objects are in an inconsistent state
(e.g. before defining the course's bibliography), but this
inconsistent state will never be seen by the other users of the
application. Those users will see the fully created object only when
the operation is committed.

\subsection{Business Transaction across multiple interactions}

The model described in the previous scenario, whereas simple and easy
to implement, may not be suited for every situation. Imagine that
instead of four attributes, Course had 50 attributes. It would then be
unfeasible to ask the user to fill everything out in a single web
page. So, the logical solution would be to split the various
attributes in multiple pages, accounting for multiple user
interactions.

Let us now assume that the creation of a course is made throughout
three interactions. In the first interaction, the user selects the
course's name, in the second interaction, the user introduces the
objectives and credits, and in the final interaction, the user selects
the bibliography, thus creating the course.

At first glance, it would seem quite easy for a programmer to change
the logic programmed in the first scenario to meet the new
requirements: the programmer would simply have to split the code
performed in a single request into three smaller parts, each to be
executed in one request.

Yet, having three separate requests implies having three different
database transactions, breaking the atomicity and isolation of the
operation. After handling the first request, the persisted domain
would be in an inconsistent state (a course with no attributes other
than its name).

The implementation of the business logic must then take this issue
into account, because the programmer cannot write the updates directly
on the domain.

This scenario represents what was defined as a Long-Lived Transaction,
in which the Business Operation has a larger lifetime that a single
database transaction (in this particular case, three database
transactions).

\section{Why are they difficult to implement?}
\label{sec:difficult}

In this multiple interaction scenario, the programmer must take
special care when implementing the operation. As mentioned above, we
cannot simply split the code used in the first scenario.

The programmer has several choices for the implementation:

\begin{enumerate}
\item Keep a database transaction open during all the steps of the
  operation.

\item Create a parallel representation of the objects that are being
  manipulated, store them outside the domain and apply the changes
  only in the last interaction.

\item Change the domain model, to represent the consistency state of
  the objects being manipulated. This affects the code that operates
  on that portion of the domain, because it must filter out objects
  that are still in an inconsistent state.
\end{enumerate}

It should be clear that any of those solutions is far from trivial to
implement, has some serious consequences (presented below), and, as I
will demonstrate, is an unnecessary burden to the programmer.

\subsection{Keeping a database transaction open}

Given that atomicity and isolation are broken due to the fact that
each interaction with the user is done within its own database
transaction, we could think the solution would be to keep the database
transaction open during the whole business transaction.

However, most modern Relational Database Management Systems (RDBMSs)
do not cope well with transactions that are open for arbitrarily large
periods of time, because a long lived database transaction may limit
concurrency, cause timeouts and deadlocks or starve the database's
connection pool. All these factors contribute to making this approach
highly undesirable, making the programmer seek alternative approaches.

\subsection{Parallel Representation of the domain}

In this approach, a series of objects similar to the domain objects
are created, and must be managed manually, kept in the user's session
context, outside the domain.

As the complexity of the domain, and the number of objects manipulated
by the operation, grow, it becomes harder for the programmer to
manually manage the data that must be kept. Ultimately, there is a
copy of the whole domain stored in the user's session, waiting for the
last transaction to transactionally update the domain with all the
information input by the user.

This is the opposite of what would be desirable for the programmer:
she should be able to operate directly on the domain.

\subsection{Changing the domain model}

Now imagine that an additional requirement is added to the system: the
user may be allowed to log off in the middle of the process, and, upon
logging back in, continue the work where she left off. Additionally,
anyone else in the user's department should be able to pick up where
the original user left off. With these requirements, it becomes clear
that simply storing a copy of the domain in session-local storage is
not enough.

Web application containers already provide an application context,
which could potentially solve this issue. Unfortunately, there are no
real guarantees that the data stored in that context will be kept for
an arbitrarily large period of time (recall that our transaction may
span several days or weeks), meaning that the intermediate data must
be kept persistently, forcing the programmer to be concerned with this
issue.

A possible solution for this requirement is to change our domain
model, by adding a new attribute to the objects being modified (in
this case, Course, as shown in
Figure~\ref{fig:courseDomainState}). The {\bf status} attribute
indicates whether the course is in a consistent state (Published) or
not (Draft). Adding this attribute has a cost: not only the domain
becomes polluted with information that is not relevant to the object
being modeled, but this solution affects other functional code across
the application, because course listings must filter out courses that
are still in the Draft status, scattering the filtering code
throughout the application.

\begin{figure}
  \centering
  \begin{tikzpicture}

\begin{class}[text width=4.5cm]{Course}{0.5,0}
  \attribute{name : String} \attribute{objectives : String}
  \attribute{credits : int} \attribute{bibliography : Publication[]}
  \attribute{status : \{Draft, Published\}}
\end{class}

\begin{class}[text width=3cm]{Department}{7.5,-1}
\end{class}

\association{Department}{}{1}{Course}{}{*}

\end{tikzpicture}

\caption{Domain Model with state representation}
\label{fig:courseDomainState}

\end{figure}

\subsection{Other approaches}

There are other approaches to Long-Lived Transactions, such as using
the Database. There are some DBMSs that announce support for
Long-Lived Transactions, such as Oracle's Workspace Manager. However,
such support is not standardized, their API's are proprietary,
rendering Web frameworks unusable in those cases (they generally only
support the SQL standard), forcing the application to be written
against the proprietary API, making it very hard to change and
maintain. All these factors contribute to make this approach
inadmissible for most cases.

\section{Applications of LLTs}

Even though the example I just presented was targeted at a system with
a rich domain model, there are several other applications of
Long-Lived Transactions.

\subsection{Database Management Systems}

The example provided above is targeted at applications with a rich
domain model, but there are many applications built directly on top of
DBMSs. Those applications typically have requirements similar to those
with a rich domain model, however, they must rely directly on the DBMS
for transactional support.

\subsection{Workflow Systems}

Workflow Management Systems are another area in which Long-Lived
Transactions are interesting. Workflow processes represent business
transactions, consisting in a flow of steps or tasks, each of them
possibly performed by different people.

It is thus vital for Workflow Systems to embrace Long-Lived
Transactions as a core concept, and many approaches have been
developed in this area.

Consider the Course Creation example. Assume that each step of the
operation is executed by a different person. A Workflow Management
System could be used to implement that operation.

\section{Objectives}

All this effort makes Long-Lived Transactions a concept that is very
hard to implement, promotes bad software engineering practices
(replication, code scattering, etc), and is difficult to reason
about. My thesis is that the support for Long-Lived Transactions
should be transparent, and at an infrastructural level, relieving the
programmer from this burden.
