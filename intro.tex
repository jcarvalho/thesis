\chapter{Introduction}

For many years, enterprise applications were developed using
two-tiered architectures. In such architectures, there was typically a
mainframe with great computational power, which served requests from
thin clients. As hardware evolved over the years, so did the
development of enterprise applications. Nowadays, most applications
are developed using a three-tier architecture: Data Tier, Application
Tier and Presentation Tier. Despite this separation, most applications
still rely on the Data Tier for transactional support.

With the adoption of multicore architectures over the past few years,
{\it Software Transactional Memory} (STM) has seen many advancements.
Because data persistency is a critical requirement in enterprise
applications, STMs have been extended to collaborate with persistent
storage systems, giving birth to the concept of {\it Persistent
  Software Transactional Memory} \cite{fernandes2011strict}. Thus,
several enterprise applications, such as the FenixEDU web application,
are now using PSTMs for transactional support.

Long-Lived Transactions were first described in 1981 as ``[..]
transactions with lifetimes of a few days or
weeks''\cite{gray1981transaction}, and can be found in many enterprise
applications. Due to their duration, Long-Lived Transactions pose some
challenges not encountered in short transactions, and thus, many
attempts have been made to support them.

My thesis consists on making Long-Lived Transactions easier to develop
on top of a Persistent STM, by providing infrastructural-level
support, relieving the programmer from all the effort that is required
to implement them.

In this thesis, I aim at adding support for Long-Lived Transactions in
applications with Rich Domain Models.  A Rich Domain Model is ``An
object model of the domain that incorporates both behaviour and
data.``, as described by \cite{fowler2003patterns}, meaning that
domain objects hold both their data and the business logic that
manipulates them, as opposed to an Anemic Domain Model, in which the
domain objects simply contain data, and the business operations are
handled by a separate Service Layer. Also, in a Rich Domain Model,
each domain object typically has an arbitrarily complex web of
associations, multivalued attributes, inheritance, and is typically a
part of some Object-Oriented design patterns.

My main target throughout this project are enterprise applications in
which the domain objects are persistent, transactionally updated, and
handled transparently at an infrastructural level (meaning that the
programmer should be mostly unaware of the persistence/data tier).

This document is organized as follows. In Section 2, I will present
Long-Lived Transactions in more detail, as well as explaining why they
are difficult to implement without infrastructural support. In Section
3, I present existing work on Long-Lived Transaction implementation
and support, as well as explain why they are not suited for our
requirements. In Section 4, I present the architecture of my solution,
and describe the Fenix Framework, which will be the base for this
work. In Section 5, I explain how the implementation will be
evaluated. 

Finally, in Section 6, I shall draw some conclusions.


% Local Variables:
% mode: latex
% TeX-master: "thesis"
% End: