\chapter{Related Work}
\label{chap:related}

In this section I describe the various areas in which an attempt to
solve the problem of Long Lived Transactions has been made, namely
Database Management Systems, Workflow Systems, and Object-Relational
Mapping Systems.

Also, due to its relevance regarding the solution described in section
4, I briefly present Software Transactional Memories (STM), Nested
Transactions, Persistent STMs, and how they cope with short lived
transactions.

\section{Transactions in Database Management Systems}
\label{sec:rdbms}

Transactions are an age-old notion in the database world. They
comprise a {\it Unit of Work} performed within a Database Management
System, and ensure that concurrent access to shared mutable data is
done in a consistent way.

By definition, a database transaction must provide the four ACID
properties:

\begin{itemize}

\item {\bf Atomicity} Requires that in a transaction, all database
  operations either {\it all} occur, or {\it none} occurs. This
  property guarantees that the transaction is indivisible and
  irreductible.

\item {\bf Consistency} Requires that every transaction will bring the
  database from one valid state to another (according to a series of
  consistency predicates defined for the database).

\item {\bf Isolation} Requires that individual record updates within a
  transaction are not visible outside of that transaction. When the
  transaction commits, all record updates are visible to the rest of
  the system.

\item {\bf Durability} Requires that once a transaction is committed,
  it will survive permanently. This means that once a transaction is
  committed, its effects on the data will be permanent.

\end{itemize}

\subsection{Isolation Levels}
\label{sec:isolation}

To better understand the desired isolation level of a Long-Lived
Transaction support system, an analysis of the most common isolation
levels is in order.

Attaining full isolation between transactions, while seemingly
necessary, can oftentimes be a burden to the DBMS, due to the required
overhead. Thus, isolation is, out of the four ACID properties, the one
most often relaxed. Isolation is typically implemented using Locks or
Multiversion Concurrency Control, which may result in a loss of
concurrency. To cope with that fact, several standard isolation levels
have been defined.

In the ANSI/ISO SQL standard\cite{melton1992ansi}, four isolation
levels have been defined (from most to least relaxed):

\begin{itemize}
\item {\bf Read Uncommitted} This is the lowest isolation level
  provided by any transaction support system. At this level {\it Dirty
    reads} may occur. A {\it Dirty read} is defined as a read
  operation that reads data updated by a live (not yet committed)
  transaction.
\item {\bf Read Committed} In this level, {\it Dirty reads} cannot
  happen (meaning they only read data written by already committed
  transactions). However, multiple reads to the same data location may
  read different values (from transactions that committed in between),
  even within a single transaction.
\item With {\bf Repeatable Read}, all reads within the same
  transaction read the snapshot established by the first read.
\item {\bf Serializable} is the {\it highest} isolation level defined
  in SQL. Serializability requires repeatable reads, as well as
  guaranteeing that {\it Phantom reads} do not occur. A {\it Phantom
    read} occurs when, in the course of a transaction, two identical
  queries (that act upon a range of values) return a different result
  collection. This can occur when running queries like ``SELECT * FROM
  users WHERE age BETWEEN 10 AND 30;''. If an interleaving transaction
  creates a new user with an age between 10 and 30 years, the second
  run of the query will show a result which contains the new user.
\end{itemize}

In \cite{papadimitriou1979serializability}, an extension to
Serializability is described: Strict Serializability. In addition to
the requirements for Serializability, Strict Serializability requires
that transaction histories are real-time ordered, meaning that all
transactions histories must be ordered in a way consistent with the
precedence order of the operations. In practice, this isolation level
is not implemented by DBMSs, due to its performance impact.

Another common isolation level (not described in SQL-92) is {\bf
  Snapshot Isolation}. In snapshot isolation, all reads made within a
transaction are guaranteed to see a consistent snapshot of the
database, and the transaction itself will only commit if no updates it
has made conflict with any concurrent updates since the
snapshot. Snapshot isolation arose from work on Multiversion
Concurrency Control, and thus did not make it into the lock-centered
mindset of the SQL-92 standard. This gap was heavily criticised in
\cite{berenson1995critique}.

\subsection{Concurrency Control}

Concurrency Control is the mechanism that ensures that ``Concurrent
execution should not cause application programs to malfunction''
\cite{reuter1993transaction}. This property was coined in 1993 as the
first law of Concurrency Control, by Jim Gray. In fact, concurrency
control mechanisms are what ensure that the ACID properties of
transactions are kept, in an environment with concurrent access to
shared mutable data. Research on the topic dates back to the 1970s
\cite{rosenkrantz1978system} and 1980s \cite{gray1981transaction}, and
is still a hot topic nowadays.

There are two main categories of Concurrency Control:

\begin{itemize}

\item {\bf Optimistic} Delaying integrity checking until the end of
  the transaction, without blocking any of its reads and writes.
  Optimistic concurrency control allows for greater concurrency,
  because concurrent operations can proceed without interfering with
  each other. On the other hand, given that integrity is only checked
  at the end of the transaction, conflicts are not detected until all
  the work is done and the transaction has to be restarted. It is
  typically implemented by Multiversion Concurrency Control and
  Timestamp Ordering mechanisms \cite{Bernstein1981}.

\item {\bf Pessimistic} Every data access in the transaction acquires
  a lock before proceeding. During the acquisition process, if an
  integrity violation is detected, the transaction is aborted, rolling
  back every write and releasing every lock held. Once the transaction
  is finished, it is marked as committed, and all the locks are
  released. The main method in this category is Two-Phase Locking
  \cite{Bernstein1981}.

\end{itemize}

Historically, pessimistic concurrency control has been the dominant
category, and even nowadays, most Relational Databases implement it.

Due to this fact, most of the work regarding Long-Lived Transaction
support in DBMSs is related with locking approaches. In fact, one of
the main reasons why databases do not cope well with long
transactions, is that the transaction holds a lock for each accessed
record. In lock-based DBMSs, the frequency of deadlock goes up with
the square of the concurrency level and the fourth power of the
transaction size \cite{gray1981transaction}. Thus, several strategies
have been devised to minimize the probability of deadlocks on
Long-Lived Transactions.

\subsection{Relaxation of Transactional Properties}

Given that holding the locks for the duration of the transaction was
not viable, the first proposed solution for this problem was to accept
a lower degree of consistency, by allowing transactions to release
their locks before commit.

In \cite{gray1981transaction}, the author proposed a model in which
only {\it active} transactions (the ones currently updating the
database) hold locks. {\it Sleeping} transactions (not currently
updating the database - also known as {\it User Think-Time}) do not
hold any locks. This means that the isolation property is broken, due
to the fact that other transactions will be able to see an uncommitted
value.

In \cite{salem1989altruistic}, the authors propose an extension to
Two-Phase locking, called {\it altruistic locking}. This extension
provides the concurrency control manager with a set of rules that
allow it to release locks early in the transaction. The altruistic
protocol takes advantage of the transaction's data access pattern. In
particular, it takes advantage of the knowledge that a transaction no
longer needs access to a data object that it has locked. Transactions
can then issue {\it release} operations, indicating that a certain
piece of data will no longer be accessed. Other transactions will be
allowed to access this data concurrently, while agreeing to abide by
certain restrictions that were put in place to ensure that all
transactions see a consistent database state.

\subsection{Transaction Logs}

Another problem with Long-Lived Transactions in DBMSs results from the
use of a Transaction Log, which contains a record of every operation
performed in the database, so that the DBMS can perform crash
recovery. The problem is that the size of that log is finite and, over
time, it fills up, thus forcing Long-Lived Transactions to abort, as
some of their log records will be overwritten by more recent
transactions.

In \cite{hagmann1991implementing}, the authors propose ``Log Record
Forwarding'', as a means to relocate log entries belonging to
Long-Lived Transactions, so that they will not be aborted for that
reason. In this proposal, active records in the log area are copied
(forwarded) to the end of the log, thus surviving another
log-reclaiming cycle.

\subsection{Sagas}
\label{sec:sagas}

Garcia-Molina and K. Salem proposed, in their 1987 paper
\cite{garcia1987sagas}, the notion of sagas. The basic idea of the
saga model is to allow a transaction to release resources prior to
committing. A Long-Lived Transaction is a saga if it can be seen as a
sequence of sub-transactions that can be interleaved in any way with
other transactions. Each sub-transaction in the saga guarantees that
the ACID properties on the database are preserved.

Partial executions of the saga are undesirable, so the DBMS is
responsible for guaranteeing that either all transactions in a saga
are successfully completed, or compensation actions are run to amend
the partial execution. Thus, each sub-transaction is associated with a
compensating transaction, which undoes the changes performed by the
original transaction. Note that this action does not necessarily
return the database to its original state, but instead acts upon the
{\it business state} of the application.

More formally, consider a saga $T$, consisting of sub-transactions
$T_1, T_2, \ldots, T_n$, and compensating actions $CT_1, CT_2, \ldots,
CT_n$. The system guarantees that either the sequence $T_1, T_2,
\ldots, T_n$ is executed (meaning the whole saga is executed), or
$T_1, T_2, \ldots, T_j,CT_j, \ldots, CT_2, CT_1$ for some $0 \le j \le
n$, will be executed (meaning that a failure occurred at $T_{j+1}$,
causing the previously executed sub-transactions to be compensated).

\subsection{Offline Concurrency Patterns}

Martin Fowler, in his book {\it Patterns of Enterprise Application
  Architecture} \cite{fowler2003patterns}, introduces a series of {\it
  Offline Concurrency Patterns}, as the building blocks for Long-Lived
Transaction support at the application level. In his patterns, Fowler
assumes that a Long-Lived Transaction is run as a series of
short-lived system transactions, whose transactional support is
provided by the underlying DBMS.

The patterns provide a workaround for the fact that the ACID
properties are broken when using several system transactions.

\begin{itemize}
\item {\bf Optimistic Offline Lock} This pattern solves the problem by
  validating that the changes about to be committed by one session (or
  business transaction) do not conflict with the changes of another
  session. This is achieved by, once the session is being committed,
  validating that the changes about to be committed are consistent,
  making sure that the value previously read by the current session
  was not changed by another session (i.e. obtaining an Optimistic
  Offline Lock).  The most common implementation of this pattern
  consists on associating a version number with the record in the
  system. When a record is loaded, the record's version is kept in the
  session, and that version is used when acquiring the Optimistic
  Offline Lock (by comparing the two versions). Once the lock is
  successfully acquired, the record's version is incremented, and the
  system transaction can be committed. If the lock cannot be acquired,
  a conflict is detected, the current system transaction is rolled
  back, and the business transaction must either abort or attempt to
  resolve the conflict and retry.

\item {\bf Pessimistic Offline Lock} This pattern prevents conflicts
  by avoiding them altogether. It forces a business transaction to
  acquire a lock in each record before using it, so once you begin the
  business transaction, you are mostly sure that it will complete
  without concurrency control issues. These locks however, are not
  database-level locks, but are instead high-level, typically managed
  by a lock manager in the application, and can have various
  granularities (which will have different impacts on system
  performance). In this pattern, whenever a business transaction
  wishes to access a record (for either read or write), it must
  acquire a lock with the lock manager, ensuring that a business
  transaction that wishes to write for that particular record is the
  only one doing so.
\end{itemize}

Typically {\bf Optimistic Offline Lock}s are used in an environment
where conflicts are expected to be low. If conflicts are likely, it is
not good user experience to report a conflict after the user finished
his work and is ready to commit. In that case, a {\bf Pessimistic
  Offline Lock} is in order, avoiding throwing away work, because
conflicts can be detected early in the transaction. Also, if the cost
of a conflict is too high, no matter its likelihood, this is the
appropriate pattern.

Although with the use of these patterns the development of Long-Lived
Transactions is facilitated, they require the programmer to pollute
the domain with the addition of code that should be at an
infrastructural level.

\section{Workflow Management Systems}

A Workflow Process is described as a sequence of connected steps (or
activities), with special emphasis on the flow paradigm, where each
step of the process follows the precedent, and ends just before the
next step can begin.

A Workflow Process typically may have a long duration, involve many
users (in fact, each step is typically executed by a different user),
and range across a variety of heterogeneous systems. Individual
activities range from computer programs and applications to human
activities such as meetings and decision making.

In this document I shall use an abstract model to describe Workflow
Processes, using the following concepts:
\begin{itemize}
\item {\bf Process} is a sequence of steps necessary to achieve a
  goal. A process consists of activities and data.
\item {\bf Activity} is each step within a process. Activities have a
  name, pre and post-conditions, and a series of constraints.
\item {\bf Control Flow} is the order in which activities are
  executed.
\item {\bf Data Flow} is the mapping between the inputs/outputs of the
  data required by the activities.
\item {\bf Conditions} specify the circumstances in which certain
  events will happen. {\it Transition Conditions} specify whether a
  certain activity may or may not become {\it startable}. {\it Start
    Conditions} specify when an activity is actually started, either
  by all its triggering events (AND) or by just one (OR). {\it Exit
    Conditions} specify whether an activity has finished successfully.
\end{itemize}
In this model, an activity with no incoming activities is considered a
{\it Start Activity}. A process is considered finished once all its
activities are terminated.

Over the years, many Workflow Modelling Languages have been devised,
and in fact, it is still a very active area of research. Examples of
such languages are YAWL \cite{van2005yawl}, AWGL
\cite{fahringer2005specification} and UEML \cite{vernadat2002ueml}, among
many others.

In \cite{sheth1993transactional}, the term {\it Transactional
  Workflow} was introduced, recognising the relevance of transactions
to workflows. Transactional Workflows provide functionality required
by each workflow process (such as task collaboration), which is
usually not available in typical DBMSs. This, however, does not imply
that workflows are similar to database transactions, nor that they
support all the ACID properties (namely concurrency control, backward
recovery, and data consistency).

Workflow Management Systems (WFMSs) are the systems that manage
user-defined Workflow processes for different types of jobs and
processes. It is the WFMS that provides the transactional properties
to the transactional workflow. These include transaction management
techniques such as logging, compensation, etc.

Given that their support for recovery and failure handling is lacking,
unlike regular transaction systems, they typically provide the means
for the user to determine the actions necessary to recover from
failure and consistency issues.

\subsection{Implementing Transactional Workflows}

In \cite{dayal1990organizing}, a workflow is seen as a {\it
  Long-Running Activity}, and is modeled as a set of execution units
that may consist recursively of other activities, or top-level
transactions. Control and data flow is specified either statically in
the activity's {\it script}, or dynamically using
Event-Condition-Action (ECA) rules. This model contemplates
compensation, execution unit communication, activity status queries
and exception handling.

In \cite{weikum1993extending}, the author suggests that semantic
transaction concepts be merged with workflow concepts to promote
consistent and reliable workflow systems. The author defines a
transactional workflow to be a control sphere that binds these
transactions, using dependencies to enforce as much behavioral
consistency as possible.

For a more comprehensive review on the topic, refer to
\cite{worah1997transactions}.

\subsection{Sagas}

The concept of Sagas was previously presented in
Section~\ref{sec:sagas}. In this section, I show how sagas can be
implemented on top of Workflow Management Systems.

All the sub-transactions of the saga are grouped into a block (Forward
Block). The control flow within the block represents that of the saga,
with each sub-transaction represented as an activity. Every activity
has an incoming transition condition, which is that the previous
activity has finished successfully (i.e. the corresponding transaction
has committed).

If a transaction aborts, the corresponding outgoing condition will
evaluate to false, and no other activity in the block will
execute. When execution of the block terminates, its data container
will contain a list of the status of each activity. In this case,
control is passed to a second block (Compensation Block), containing
the compensating activities in reverse order.

When the compensation block is executed, all the conditions that
correspond to activities that executed in the Forward Block are
activated, causing the compensating activities to be executed in
order.

\subsection{Implementing Long-Lived Transactions using WFMSs}

As described in the previous sections, several attempts have been made
to implement Advanced Transaction Models on top of Workflow Systems,
as a means of supporting Long-Lived Transactions.

Several other methods have been proposed in the literature (e.g.,
\cite{alonso1996advanced, 798492}). However, such implementations are
not suitable for the requirements defined in Section~\ref{sec:what},
due to the fact that most WFMSs do not provide full ACID properties.

\section{Object-Relational Mapping}
\label{sec:orm}

Due to the increase in popularity of object-oriented programming, a
large number of Object-Relational Mapping (ORM) tools have arisen over
the past few years \cite{orm}. These tools range from simple data
access layer abstractions (which simply abstract the necessary
instructions to perform the desired operation), to fully fledged
tools that handle every aspect of the Object-Relational Mapping,
making the underlying database completely transparent to the
programmer.

Object-Relational Mapping tools are available for a great number of
languages. For Java, there are many competing persistence standards,
including: Java Persistence
API\footnote{\texttt{http://jcp.org/en/jsr/detail?id=317}} (JPA), Java
Data
Objects\footnote{\texttt{http://www.jcp.org/en/jsr/detail?id=243}}
(JDO), Spring
Data\footnote{\texttt{http://projects.spring.io/spring-data/}}, among
others. Both specifications allow a programmer to create domain
entities by decorating Plain Old Java Objects (POJOs) with
annotations, making the persistence layer transparent to the
programmer. Whereas JDO provides a larger number of features than JPA,
the latter has seen greater adoption by the community, both in terms
of usage and support. The most well-known implementations of the JPA
are Hibernate (\texttt{http://www.hibernate.org}), OpenJPA
(\texttt{http://openjpa.apache.org}) and EclipseLink
(\texttt{http://www.eclipse.org/eclipselink/}).

To provide transactional support for domain objects, ORMs typically
delegate transaction management to the underlying Database (which is
assumed to provide full ACID support). In fact, as of this writing,
all the examples shown above operate this way. Thus, Long-Lived
Transaction support in ORM tools is largely dependent on the
underlying DBMS, presenting the same challenges as described in
Section~\ref{sec:rdbms}.

Given the popularity of Java, I will now consider the JPA, and one of
its most popular implementations, Hibernate, to understand how they
cope with Long Lived Transactions. Note that despite the concreteness
of the examples presented, their concepts are generic enough to be
applied to any other ORM tool.

\subsection{JPA Optimistic Concurrency Control}

The specification of the Java Persistence API assumes the use of
optimistic concurrency control (or optimistic locking). In this
context, optimistic locking is used to prevent the database from
holding on to critical resources, potentially causing high degrees of
contention.  This is achieved by operating directly on the domain
objects (in memory), delaying the propagation of the changes to the
database as much as possible.

As part of the JPA, there is the concept of Version field. This field
allows {\it disconnected operation}, meaning that the reads/writes
from the database are deferred until a checkpoint or the end of
transaction. When using this optimistic approach, all the versions of
the objects used in the transaction are checked against the version
present in the database, thus ensuring that no dirty reads will occur.

\subsection{Hibernate Long Conversations}

Hibernate supports the concept of {\it Long Conversations} by allowing
a session\footnote{A Hibernate session implements the Unit of Work
  design pattern \cite{fowler2003patterns}.} to remain open as long as
the user interaction lasts, potentially executing multiple database
transactions. However, the programmer is responsible for ensuring the
atomicity and isolation of the business transaction, by following a
very strict pattern: All transactions but the last must only read
data, and, thus only the last database transaction can update
data. Hibernate assists the programmer in verifying that the data read
across the multiple transactions is consistent, using an object
versioning mechanism. However, this verification only works if all the
data required for the application is read within the first database
transaction.

This support is not good enough for several reasons:

\begin{itemize}
\item It imposes a pattern that may not be suited for many
  applications.
\item All the state of the transaction is kept in memory, which may
  cause the application to run out of memory in very large
  transactions.
\item The session is not kept persistent. If the application server
  restarts in the middle of the transaction, all data related to the
  ongoing Long Conversation is lost.
\item Despite the versioning support, the business transaction may
  still suffer from inconsistent reads, as it spans multiple isolated
  database transactions. Suppose that in a database transaction (T2)
  other than the first one (T1), the application reads an object for
  the first time. If the object was concurrently modified by yet
  another transaction (T3) that occurred between T1 and T2, T2 would
  see the value written by T3, which would not be consistent. As such,
  these transactions are not {\it Serializable} (See Section \ref{sec:isolation}).
\end{itemize}

All of these reasons violate the desired properties described in
\ref{sec:difficult}, making this system undesirable.

\section{Software Transactional Memories}
\label{sec:stm}

Given that the solution I propose in Section \ref{sec:arch} is built
upon a Software Transactional Memory (STM), I provide here a brief
introduction to the topic.

The original idea behind Transactional Memories (TM)
\cite{herlihy1993transactional} is to provide efficient lock-free data
structures for highly concurrent systems using hardware. Its main goal
was to free programmers from using locks and monitors to manage
concurrent accesses to shared data. Managing lock granularity and
composability is a cumbersome and error-prone task, and may lead to
unwanted situations such as deadlocks. Over the years, the original
concept of Hardware Transactional Memories has evolved, giving birth
to Software Transactional Memories \cite{shavit1997software}.

Software Transactional Memories bring into the realm of programming
languages, the age-old notion of transactions, well-known in the area
of DBMSs. However, unlike those, STMs are not concerned with the
Durability property of the ACID model, and thus, many STM
implementations have little in common with their database
counterparts.

There have been several recent proposals for STM implementations, such
as those described in \cite{cachopo2006versioned, herlihy2003software,
  marathe2005adaptive, dice2006transactional, riegel2006lazy,
  marathe2006lowering}.

STMs use transactions to isolate memory operations in atomic Units of
Work. A transaction typically contains a Read Set and a Write Set,
which are used to register its read and write operations,
respectively. The STM provides a series of mutable memory locations
that can be read from or written to. In many STMs, it is up to the
programmer to confine all of the application's mutable state to these
memory locations, whereas some STMs provide fully transactional
memory. It is the responsibility of the STM to guarantee that
concurrent accesses to such memory locations are correct according to
the specified correctness criteria.

\subsection{Correctness Criteria}
\label{sec:opacity}

In Section~\ref{sec:isolation}, I presented the typical isolation
levels (or correctness criteria) that can be found in DBMS. In
\cite{guerraoui2008correctness}, the author claims that the {\it
  traditional} correctness criteria are unfit for TM implementations,
and proposes the {\it opacity} criterion. Opacity is described as a
safety property that ensures that ``[...] (1) all operations performed
by every {\it committed} transaction appear as if they happened at
some single, indivisible point during the transaction lifetime, (2) no
operation performed by any {\it aborted} transaction is ever visible
to other transactions (including live ones), and (3) every transaction
always observes a {\it consistent} state of the system.'', meaning
that unlike Serializability, even {\it non-committed} transactions are
prevented from accessing inconsistent states. This correctness
criteria is ensured by many TM systems.

\subsection{Nested Transactions}

The best practices in software development encourage programmers to
modularize their applications and have them abide by well-defined
interfaces. Lock-based approaches do not cope well with this
principle, because the programmer must be aware of module's internal
locking conventions. Transactions, on the other hand, do not have this
limitation, allowing for better composability.

In the transaction paradigm, it would then be common for application
code to make a call to a library that uses transactions to protect its
shared internal state. In this scenario, the transaction created by
the library would be {\it nested} in the outer (application-owned)
transaction. Nested Transactions are an extension to the basic
transaction structure, supporting multiple levels of transactions:
Transactions that may contain other sub-transactions, forming a
transaction tree. Nested Transactions have long been used in the
database world, and many of its concepts have been adapted to
transactional memories.

Consider the kind of monolithic transactions commonly found in
enterprise applications. In such transactions, the volume of work that
must be done if the transaction needs to be rolled-back is
increased. Using nested transactions, smaller portions of the work can
be done in isolation, each within a nested transaction. If one of
those portions fails, only that portion needs to be rolled-back, and
an attempt can be made to compensate for that failure. Similarly,
clients of opaque libraries may suffer performance hits if the
transaction fails because of the work performed in said libraries.

In \cite{moss2006nested}, Moss and Hosking defined two types of
nesting: Closed and Open.

\subsubsection{Closed Nesting}

A transaction is either {\it top-level}, which is similar to a
non-nested transaction, or is nested within a {\it parent}
transaction. In this model, only transactions with no current children
can access data. Transactions accumulate Read and Write sets, which
will determine the outcome of the transaction. When a transaction
reads a value, it sees either the value in its own read or write set,
or the value seen by its parent (a top-level transaction sees the
latest value).

When a nested transaction commits, its read and write sets are merged
with the ones of its parent. It is only when the top-level transaction
commits that its writes become permanent. If the nested transaction
aborts, its read and write sets are discarded.

\subsubsection{Open Nesting}

Open Nested Transactions relax the isolation requirements by making
the results of committed sub-transactions visible to other
concurrently running transactions. When a sub-transaction commits,
instead of merging the write set with its parent's, the writes are
committed at the top-level. This way, a higher degree of concurrency
is achieved, because resources may be released earlier and conflict
detection can be applied at a higher level. It is up to the
implementations to provide the mechanisms to handle recording of
necessary compensating actions in case an ancestor transaction aborts.

\subsection{Nested and Long-Lived Transactions}

Nested Transactions present many similarities with Long-Lived
Transactions. In both cases there is a top-level transaction, which
can be divided in several smaller steps. Open nesting uses the same
basic principle as the techniques described in previous sections,
based on the relaxation of transactional properties, and compensating
actions. Closed Nesting is similar to the desired properties presented
in Section~\ref{sec:what}.

\section{Persistent STMs}
\label{sec:pstm}

Over the years, Enterprise Application architecture has evolved at a
very fast pace. However, one aspect has remained constant: They still
rely on a relational database to handle both data persistence and
transactional support. In \cite{fernandes2011strict}, the authors
argue that such design, while once justified by hardware limitations,
is no longer suited for today modern applications.

A new architecture was proposed, using a Software Transactional Memory
for transaction support at the application server tier, shifting the
responsibility from the DBMS to the application server. The database
still plays an important role in this architecture, due to the fact
that the STM must be extended to support persistence (PSTMs).

By shifting the responsibility for transaction handling to the STM,
enterprise applications can benefit from the many advancements in the
STM area, such as multi-core machine scaling, stronger correctness
guarantees and nonblocking progress conditions such as lock freedom
\cite{fernandes2011lock}.

Using a PSTM, enterprise application developers no longer have to
trade correctness for performance. Modern PSTM implementations allow
for Strict Serializability, while providing better performance than
regular DBMSs, yielding a throughput increase of up to 23 times
\cite{fernandes2011strict}.

Another advantage of PSTMs is that, like in STMs, the programmer
benefits from a much simpler and transparent programming model,
allowing for much cleaner and maintainable code, better composability
by using lock-free data structures, making the process of translating
business requirements to code much simpler and bug-free.

In this project, I aim at extending the PSTM model, by adding the
necessary building blocks to support Long-Lived Transactions
(described in Section~\ref{sec:arch}).

\section{Why those solutions are unfit}

Among all the solutions presented above, one common aspect stands out:
They all require the programmer to deal with Long-Lived Transactions
at the application level, which goes against my claim that support
should be transparent, and at an infrastructural level.

Some of the presented solutions are based on relaxation of
transactional properties, such as allowing breaches in
isolation. Long-Lived Transactions should run with the same
transactional properties as regular transactions, and thus, relaxation
is not acceptable.

Despite the similarities between Workflow Processes and Long-Lived
Transactions, WFMSs are not suited for our requirements. Typical WFMSs
are not concerned with the ACID properties of transactions, and rely
heavily on high-level compensatory actions for data consistency.

The solution proposed in Section~\ref{sec:orm} (Hibernate) imposes a
very strict programming model, which may not be adequate for some
applications, and does not guarantee that Long-Lived Transactions will
have the same transactional properties as regular transactions.

None of the proposed solutions fit the requirements specified in
Chapter 2, and thus, a new approach is needed.
