\chapter{Fenix Framework}
\label{chap:ff}

``Fenix Framework allows the development of Java-based applications
that need a {\it transactional and persistent domain model.}''

This chapter describes in detail the major components of the Fenix
Framework, used to implement the solution proposed throughout this
document. Section \ref{sec:dml} describes the Domain Modelling
Language, used to describe the application's domain model. Section
\ref{sec:ff-arch} describes the high-level architecture of the
Framework, briefly describing its major components and their
interaction. Section \ref{sec:jvstm} presents the Java Versioned STM
(JVSTM), and its integration with the Fenix Framework. The information
presented in this chapter is critical to understanding the proposed
solution, as well as its challenges.

\section{Domain Modelling Language}
\label{sec:dml}

The Fenix Framework is aimed at entreprise-class applications with a
rich domain model in an object-oriented paradigm. Such applications
typically consist of class hierarchies representing entities with
relationships among them, forming an interconnected graph. 

The Domain Modelling Language (DML) is a Domain-Specific Language
designed to represent such domain models, separating the domain's
structure from its behaviour. The DML is designed with modularity as a
core concern, allowing for incremental and modular domain definition.

In a DML file, programmers write their domain definition in a
Java-like language. A class definition consists of the class name, the
persistent slots (either primitive or value types), and the
super-class. Listing \ref{list:class-example} shows how the {\it
  Course} class from Figure \ref{fig:courseDomain} could be described
in the DML. Note that as arrays are not natively supported, a Value
Type must be created, describing an array of publications. Value Types
are described in more detail below.

\begin{lstlisting}[caption={DML for the {\it Course} class},
  label={list:class-example}]
class Course {
  String name;
  String objectives;
  int credits;
  PublicationList bibliography;
}
\end{lstlisting}

Relations in DML are named, first-class entities. A 
 

\subsection{ValueTypes}
\subsection{JSON}


\section{Architecture}
\label{sec:ff-arch}


\section{JVSTM}
\label{sec:jvstm}

\subsection{VBoxes}

\subsection{Integration}

\subsection{To-Many Relations}

