\thispagestyle{empty}

\vbox to\textheight{

\vbox to3cm{
\includegraphics[width=5cm]{IST_A_CMYK_POS}
}

% imagem (opcional)
\vbox to5cm{

}


\vskip6mm
\vbox to25mm{
\fontsize{16}{16}\selectfont\bf
\vfil
\begin{center}
Making Long-Lived Transactions Easier to Develop
\end{center}
\vfil
}

\vskip10mm
\vbox to25mm{
\vfil
\begin{center}
\fontsize{14}{14}\selectfont\bf Jo\~ao Pedro Garcia Franco Carvalho\\
\end{center}
\vfil
}

\vskip8mm

\vbox to8mm{\fontsize{12}{12}\selectfont
\vfil
\centerline{Disserta\c{c}\~ao para obten\c{c}\~ao do Grau de Mestre
  em}
\fontsize{16}{16}\selectfont
\bf\centerline{Engenharia Inform\'{a}tica e de Computadores}
\vfil
}

\vskip10mm

\vbox to7mm{
\vfil
\begin{center}
{\fontsize{14}{14}\selectfont\bf J\'{u}ri}\\
\end{center}
\vfil
}

\vbox to28mm{
\vfil
\begin{center}
\fontsize{12}{12}\selectfont
Presidente: Prof. Doutor Luís Eduardo Teixeira Rodrigues\\
Orientador: Prof. Doutor João Manuel Pinheiro Cachopo\\
Vogal: Prof. Doutor João Ricardo Viegas da Costa Seco\\
\end{center}
\vfil
}

\vskip28mm
\vbox to4mm{\fontsize{14}{14}\selectfont\bf
\vfil
\begin{center}
Outubro 2013
\end{center}
\vfil
}
}
\newpage
\thispagestyle{empty}

  %%%%%%%%%%%%%%%%%%%%%%%%%%%%%%%%%%%%%%%%%%%%%%%%%%%%%%%%%%%%%%%%%%%%%%%%%%%%%
  %
%%%%%                             AGRADECIMENTOS
 %%%
  %

\chapter*{Acknowledgements}
\thispagestyle{empty}

There are many people to thank for the development of this work, many
of them without whom it would not have been possible to complete. I
first would like to thank everyone at the ESW Sofware Engineering
Group at INESC-ID, especially my advisor, Professor João Cachopo. His
guidance, knowledge and support were a great contribution to the
development of this work, and its quality is greatly due to him.

I would also like to thank my colleages at IST's DSI, in the FenixEdu
project, whose ideas and feedback were critical to the making of this
work. A special thanks to my current colleages: Luis Cruz, Susana
Fernandes, Ricardo Rodrigues, Pedro Santos, Sérgio Silva, Artur
Ventura and David Martinho; as well as my former colleages: Diogo
Simões, João Neves and João Antunes.

A special thanks to my family and friends who ran this journey by my
side and always believed in what I was trying to accomplish.

I dedicate this work to my grandmother Veneranda, whose strength and
perserverance were crutial in completing this work.


\vfill
\begin{flushright}
  \begin{minipage}{8cm}
    \begin{center}
      Lisboa, \today

      Jo\~ao Pedro Garcia Franco Carvalho
    \end{center}
  \end{minipage}
\end{flushright}

\newpage
\thispagestyle{empty}

\cleardoublepage

  %%%%%%%%%%%%%%%%%%%%%%%%%%%%%%%%%%%%%%%%%%%%%%%%%%%%%%%%%%%%%%%%%%%%%%%%%%%%%
  %
%%%%%                            DEDICATÓRIAS
 %%%
  %

\chapter*{}
\thispagestyle{empty}

\vfill
\mbox{}
\vfill\large
\begin{flushright}
  \begin{minipage}{8cm}
    \begin{center}
      {\it Being the richest man in the cemetery doesn't matter to me. Going to bed at night saying we've done something wonderful, that's what matters to me.}
    \end{center}
    \begin{flushright}
      - Steve Jobs
    \end{flushright}
  \end{minipage}
\end{flushright}
\normalsize\vfill
\newpage
\thispagestyle{empty}

\cleardoublepage

  %%%%%%%%%%%%%%%%%%%%%%%%%%%%%%%%%%%%%%%%%%%%%%%%%%%%%%%%%%%%%%%%%%%%%%%%%%%%%
  %
%%%%%                                RESUMO
 %%%
  %


\chapter*{Resumo}
\thispagestyle{empty}

Ao longo dos últimos anos, a Memória Transacional tornou-se um tópico
bastante popular, crescendo para além de um mero tópico de
investigação.  Mais recentemente, o conceito foi extendido para
suportar persistência, e como tal, o conceito de Memória Transacional
Persistente (PSTM) foi criado. Nesta dissertação irei propor uma
extensão às PSTMs para suportar Transacções de Longa Duração. Uma
Transacção de Longa Duração é uma Transação com um tempo de vida
superior a uma Transacção simples, executada em vários passos
disjuntos. Os sistemas de suporte transacional existentes hoje em dia
não estão preparados para lidar com Transacções de Longa Duração, e
como tal, os programadores de aplicações empresariais vêm-se forçados
a utilizar {\it workarounds} para as implementar.

A minha tese é que o suporte a Transações de Longa Duração deveria ser
fornecido a nivel infra-estrutural, usando uma Memória Transacional
Persistente. Irei descrever os desafios que tornam as Transacções de
Longa Duração difíceis de implementar, e irei propor uma solução para
facilitar o seu desenvolvimento. Irei mostrar como os programadores
podem tirar partido de Transacções de Longa Duração que sobrevivem a
{\it restarts} do sistema, requerem modificações de código mínimas,
permitem vários utilizadores concorrentemente e com resultados de
performance comparáveis às transacções simples.

\newpage
\thispagestyle{empty}

\chapter*{Abstract}
\thispagestyle{empty}

Over the past years, Software Transactional Memories have become more
and more popular, growing to be something more than simply a research
topic. On top of that, the concept has been extended to encompass
persistence, so the concept of Persistent Software Transactional
Memories (PSTM) was born. In this dissertation, I propose an extension
to PSTMs to support Long Lived Transactions. Long Lived Transactions
are transactions with a lifespan larger than a typical transaction,
executed in multiple disjoint steps. Current Transaction Support
Systems do not cope well with Long Lived Transactions, forcing
programmers to devise clever ways to implement them.

My thesis is that supporting Long Lived Transactions should be done at
the infrastructural level on top of a Persistent STM.  I will describe
the challenges that make Long Lived Transactions hard to implement,
and propose a solution to address them. I show how programmers can
take advantage of Long Lived Transactions that can survive application
restarts, require minimal code modifications, allow multiple
concurrent users and show minimal overhead in relation to regular
transactions.


\newpage
\thispagestyle{empty}

  %%%%%%%%%%%%%%%%%%%%%%%%%%%%%%%%%%%%%%%%%%%%%%%%%%%%%%%%%%%%%%%%%%%%%%%%%%%%%
  %
%%%%%                 FICHA BIBLIOGRAFICA -- PALAVRAS CHAVE
 %%%                            (4 a 6 keywords)
  %

\chapter*{Palavras Chave \\ Keywords}
\thispagestyle{empty}

\section*{Palavras Chave}
{\large

\noindent Memória Transaccional

\noindent Transacções

\noindent Transacções de Longa Duração

\noindent Consistência de dados

}

\section*{Keywords}

{\large

\noindent Transactional Memory

\noindent Transactions

\noindent Long Lived Transactions

\noindent Data Consistency

}

\newpage
\thispagestyle{empty}

\cleardoublepage

  %%%%%%%%%%%%%%%%%%%%%%%%%%%%%%%%%%%%%%%%%%%%%%%%%%%%%%%%%%%%%%%%%%%%%%%%%%%%%
  %
%%%%%                             INDICES
 %%%
  %

\pagenumbering{roman}

\tableofcontents
\newpage

\listoffigures
\newpage

\pagenumbering{arabic}

  %%%%%%%%%%%%%%%%%%%%%%%%%%%%%%%%%%%%%%%%%%%%%%%%%%%%%%%%%%%%%%%%%%%%%%%%%%%%%

% Local Variables:
% mode: latex
% TeX-master: "thesis"
% End:
